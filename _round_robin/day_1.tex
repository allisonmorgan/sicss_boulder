\documentclass[11pt]{article}
\usepackage{amsmath,amssymb,amsthm}
\usepackage[colorlinks = true,
            linkcolor = blue,
            urlcolor  = blue,
            citecolor = blue,
            anchorcolor = blue]{hyperref}
\usepackage{listings}
\usepackage{textcomp}
\lstset{language=Python,upquote=true,breaklines=true}
\usepackage{graphicx}
\usepackage[margin=1in]{geometry}
\usepackage{fancyhdr}
\usepackage{wrapfig}
\usepackage{multicol}

\setlength{\parindent}{0pt}
\setlength{\parskip}{5pt plus 1pt}
\setlength{\headheight}{13.6pt}
\newcommand\question[2]{\vspace{.25in}\hrule\textbf{#1: #2}\vspace{.5em}\hrule\vspace{.10in}}
\DeclareMathOperator{\Tr}{Tr}
\renewcommand\part[1]{\vspace{.10in}\textbf{(#1)}}
\newcommand{\exedout}{%
  \rule{0.8\textwidth}{0.5\textwidth}%
}
\pagestyle{fancyplain}
\lhead{\textbf{\NAME\ }}
\chead{\textbf{SICSS @ CU Boulder - Day 1: Ethics}}
\rhead{August 13, 2018}
\begin{document}\raggedright
\newcommand\NAME{Allie Morgan} 

\vspace{-0.2 in}
\question{1}{What's your working definition or previous exposure to ethics? What do you think are the opportunities and limitations?}
With your neighbor, discuss a research scenario in which you have felt ethically unsure. 
\begin{itemize}
\item Explain why you felt unsure. 
\item How did you proceed (or how do you plan to proceed) with your research project?
\end{itemize}

\vspace{1 in}
\question{2}{Make sure that you have Anaconda, Python 3, and Jupyter notebooks installed on your machine. We will not be using either today, but will begin using both tomorrow. Work with your neighbor and ask Allie or Brian if you run into any problems.} 
\begin{itemize}
\item Installing Anaconda: \href{https://www.anaconda.com/download/}{https://www.anaconda.com/download/}
\item Check that you have Jupyter installed correctly.
	\begin{itemize}
	\item Download this notebook: \href{https://github.com/allisonmorgan/sicss\_boulder/blob/master/misc/introduction\_to\_jupyter\_notebook.ipynb}{github.com/allisonmorgan/sicss\_boulder/blob/master/misc/introduction\_to\_jupyter\_notebook.ipynb}
	\item Navigate to the location of that notebook on your machine in terminal (\texttt{cd Downloads})
	\item Type \texttt{jupyter notebook} to launch the notebook in your default browser
	\item Visit the notebook in your browser and try to run all of the cells
	\end{itemize}
\end{itemize}
\end{document}
